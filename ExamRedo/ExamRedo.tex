\documentclass[11pt, oneside]{book}   	% use "amsart" instead of "article" for AMSLaTeX format
\usepackage[margin=1in]{geometry}                		% See geometry.pdf to learn the layout options. There are lots.
\geometry{letterpaper}                   		% ... or a4paper or a5paper or ... 
%\geometry{landscape}                		% Activate for for rotated page geometry
%\usepackage[parfill]{parskip}    		% Activate to begin paragraphs with an empty line rather than an indent
\usepackage{graphicx}				% Use pdf, png, jpg, or eps§ with pdflatex; use eps in DVI mode
\graphicspath{ {images/} }							% TeX will automatically convert eps --> pdf in pdflatex
\usepackage{amssymb}
\usepackage{amsmath}
\usepackage{xcolor}
\usepackage{color}
\usepackage{qtree}

\setlength{\parindent}{0em}

\title{EECS 769 Exam Redo}
\author{Elise McEllhiney}
\date{\today}							% Activate to display a given date or no date

\begin{document}
\maketitle

\section{Problem 1}
\subsection*{Consider $M$ discrete memoryless channels, channel $k$ has input alphabet $\mathcal{X}_k$, output alphabet $\mathcal{Y}_k$, and is described by probabilities $P_k(y|x)$.  Let $C_k$ denote the capacity of channel $k$ (computed in log base e).\\\\
Consider a new channel described as follows: the inputs to the channel is a pair $(k,x)$ where $k \in \{1,...,M\}$, and $x \in \mathcal{X}_k$.  When the input is $(k,x)$, the output of the channel is $(k,y)$ where $y$ is chosen according to $P_k(y|x)$. [In other words, the user of the new channel chooses which of the $M$ original channels he wants to use with which input, and the channel output indicates which channel was used, and the output of that channel.]}
\subsubsection{(a) Show that $$I(K,X;K,Y) = H(K) + I(X;Y|K)$$}

\begin{equation}\label{}
    \begin{split}
        I(K,X;K,Y) & = I(K;K,Y|X) + I(X;K,Y|X) \qquad \text{by chain rule} \\
        & = H(K) + I(X;Y|K)
    \end{split}
\end{equation}

\subsubsection{(b) Show that $$I(K,X;K,Y) \leq H(K) + E[C_K]$$ where $E[C_K]=\sum^M_{k=1}Pr(K=k)C_k$ \\
Indicate the conditions under which the inequality will become equality.}

$$I(K,X;K,Y) \leq H(K) + E[C_K]$$
$$I(X;Y|K) + H(K) \leq H(K) + E[C_K]$$
$$I(X;Y|K) \leq E[C_K]$$

The mutual information (or what we can send on the channel) given the channel K will be less than or equal to the expected value of the capacity of the channel.  This is true because of the source-channel coding theorem that states that we can't exceed capacity.  We get equality iff we maximize this for $p_x(y|x)$, that is: $$\max_{p_x(y|x)} I(X;Y)=I(X;Y|K)$$

\subsubsection{(c) Let $p_k = Pr(K=k)$.  Show that $H(K)+E[C_K]$ is maximized if $p_k=\exp(C_k)/ \sum^M_{i=1}\exp(C_i)$.}

$H(K)$ is maximized when $X$ is a uniform distribution over the alphabet.  Since we have $k=1,2,...M$ channels we know that the alphabet size of each channel is $\exp{C_k}$.  We also know that the effective alphabet size for the combination of all the parallel channels is $\sum^M_{i=1}\exp{C_i}$.

This equation is the extended form of $\exp{C_1}+\exp{C_2}=\exp{C_{1,2}}$  If there is a third channel we then have $\exp{C_{1,2,3}}=\exp{C_{1,2}}+\exp{3}$  From the above equation we can now write:
$$\exp{C_{1,2,...M}} = \sum^M_{i=1} \exp{C_i}$$

If X is to be uniformly distributed over the entire effective alphabet, the probability of the M channels must be distributed proportional to their effective alphabet size.  Or (with $p_k$ being the probability that the channel is used):
$$p_k = \frac{\exp{C_i}}{\sum^M_{i=1}\exp{C_i}}$$

\subsubsection{(d) Show that the capacity of the channel is $$\ln{ \sum^M_{i=1}\exp(C_i)}$$}

As stated above, we know that:
$$\exp{C_K}=\sum^M_{i=1}\exp{C_i}$$
$$\therefore C_K = \ln{\sum^M_{i=1}\exp{C_i}}$$

\section{Problem 2}
\subsection*{(a) Let $X_1,X_2,$ and $Y$ be 3 random variables such that $X_1$ and $X_2$ are independent. Is:}
\subsubsection{i)   $I(X_1;Y|X_2) \leq I(X_1;Y)$; or}
\subsubsection{\colorbox{yellow}{ ii)  $I(X_1;Y|X_2) \geq I(X_1;Y)$; or} }
\subsubsection{iii) a definitive relationship can not be established }
\subsubsection{Give proof for your assertion}

$$I(X_1;Y,X_2) = I(X_1;Y,X_2)$$
$$I(X_1;Y) + I(X_1;X_2|Y) = I(X_1;X_2)+I(X_1;Y|X_2)$$
Since $X_1$ and $X_2$ are independent we know that $I(X_1;X_2)=0$ this becomes:
$$I(X_1;Y|X_2) - I(X_1;Y) = I(X_1;X_2|Y) \geq 0 \qquad \text{by property of non-negativity}$$
Therefore:
$$I(X_1;Y|X_2) \geq I(X_1;Y)$$

\subsection*{(b) We can think of $X_1$ and $X_2$ as the transmitted symbols of two non-cooperating users in a multiple access channel whereby two users are trying to communicate with a single receiver simultaneously:\\\\
Although we have not formally discussed multiple-access channels, can you venture and interpretation of your result in (a) in terms of what is says about the performance limits of communication in such channels? (Hint: $X_1$ and $X_2$ are interfering with each other)}

We will be limited in performance because of the interference.  My conclusion in part a suggests that the interference is strictly destructive and that our upper bound for capacity for either channel is the capacity of the channel without interference (i.e. if we know the other value $x_2$)



\section{Problem 3}
\subsection*{(a) A source has an alphabet of 4 letters $a_1,a_2,a_3,a_4$, and we have the condition $p(a_1) > p(a_2) = p(a_3) = p(a_4)$.  Find the smallest number $q$ such that $p(a_1) > q$ implies $n_1 = 1$, where $n_1$ throughout is the length of the code word for $a_1$ in a binary Huffman code for the source.}

$$p(a_1) \geq p(a_2) = p(a_3) = p(a_4) $$
In order for n=1 (i.e. $p(a_1)$ is never required to be removed via huffman reduction) we must have:
$$p(a_1) \geq p(a_2) \qquad \text{known to be true from the initial equation}$$
$$p(a_1) \geq p(a_3) + p(a_4)$$
$$p(a_1) \geq 2p(a_2)$$
We also know:
$$p(a_1) + p(a_2) + p(a_3) + p(a_4) = 1$$
$$p(a_1) = 1-3p(a_2) \qquad \text{since $p(a_2) = p(a_3) = p(a_4)$}$$
Combining above equations we get:
$$1-3p(a_2) \geq 2p(a_2)$$
$$1 \geq 5p(a_2)$$
$$\frac{1}{5} \geq p(a_2)$$
$$\therefore p(a_1) \geq \frac{2}{5}$$

\subsection*{(b) Show by example that if $p(a_1)=q$ [your answer in part (a)], then a Huffman code exists with $n_1 > 1$}

\Tree
[. 
	[.{3/5 (0)} 
		[.{2/5 (0)} 
			[.2/5 \textit{a_1} ] 
		]
		[.{1/5 (1)} 
			[.1/5 \textit{a_2} ] 
		]
	]
        	[.{2/5 (1)} 
        		[.2/5 
			[.{1/5 (0)} \textit{a_3} ]
			[.{1/5 (1)} \textit{a_4} ]
		]	
	]
]

\begin{tabular}{ c | c }
  a1 & 00 \\
  a2 & 01 \\
  a3 & 10 \\
  a4 & 11 \\
\end{tabular}
                                      
\subsection*{(c) Now assume the more general condition: $p(a_1) > p(a_2) \geq p(a_3) \geq p(a_4)$.  Does $p(a_1) > q$ still imply that $n_1 = 1$? Why or why not?}

Yes: $p(a_1)>0.4$ still implies that $n_1=1$.  All that matters is that $p(a_1)$ is > the other two values at the end of the Huffman Reduction.  This is true because of the following statements:
$$p(a_1) > p(a_2) \qquad \text{Given by problem}$$
$$p(a_2) > p(a_3) + p(a_4)$$
The maximum possible value for $p(a_3)+p(a_4)$ occurs when $p(a_2)=p(a_3)=p(a_4)$ and as such, $p(a_3)+p(a_4)=2/5$ so we know both conditions will hold true, and as such that $p(a_1)>2/5$ will ensure that $n_1=1$.

\subsection*{(d) Now assume that the source has an arbitrary number $K$ or letters, with $$p(a_1) > p(a_2) \geq ... \geq p(a_K)$$ Does $p(a_1) > q$ now imply that $n_1 = 1$? Explain.}

Yes: $p(a_1)>0.4$ still means that $n_1=1$ since the probabilities must always sum to 1.  If $p(a_1)=0.4$ it means that the rest of the probabilities must sum to $0.6$. The failing condition in this case is if any of the values reached during the Huffman reduction (before termination when we have 2 values remaining) sum to a value $>0.4$.  For example, if we managed to have the two lowest probabilities from a set of 4 remaining probabilities (after however many rounds of the Huffman reduction algorithm are required to achieve 4 remaining values) sum to $>0.4$.  The two smallest possible probabilities that could sum to a value greater than $0.4$ would be $0.2$ and $0.2+\delta$ where $\delta$ is arbitrarily small.  However this means that our final 4 probabilities would be $0.4, 0.2+\delta, 0.2, 0.2-\delta$.  As you can see, the smallest two probabilities of this set are $0.2$ and $0.2-\delta$.  This means that so long as $p(a_1)>p(a_2)\geq ...$ we will never be able to achieve a summation that exceeds $0.4$ while following the Huffman reduction algorithm where we always sum the two smallest probabilities.  Of course, if we start with $p(a_2)=0.4$ then when we reach the final 3 probabilities we will get to choose between the $0.4$ values since the remaining probabilities will sum to $0.2$.  This is why $p(a_1)>q$ implies $n=1$ instead of $\geq$.

\subsection*{(e) Assume $p(a_1) \geq p(a_2) \geq ... \geq p(a_K)$. Find the largest number $q'$ such that $p(a_1) < q'$ implies that $n_1 > 1$.}

$q'=0.4$.  If $p(a_1)=0.4-\delta$ then the final four values could easily achieve the values $0.4-\delta, 0.2, 0.2, 0.2+\delta$.  Therefore the final three values are $0.4-\delta, 0.4, 0.2+\delta$ as you can see, our next reduction will include $p(a_1)=0.4-\delta$ and we will have $n_1>1$ as a result.  The explanation of the converse is written out in part (d) where I explain why for any value $p(a_1)>0.4$ will result in $n_1=1$



\section{Problem 4}
\subsection*{(a) Consider the Z-channel below with input distribution $q = Pr(X=0), 1-q=Pr(X=1)$. Denote the capacity of the Z-channel as $C_Z$.  Find the value of $q$ (as a function of $\epsilon$) that achieves capacity.}

$$H(Y|X)=P(X=0)H(1-\epsilon, \epsilon) + P(X=1)H(1,0) = P(X=0)H(1-\epsilon, \epsilon) = qH(1-\epsilon, \epsilon)$$
$$H(Y)=H(\epsilon)=H(Pr(Y=0))=H(q(1-\epsilon))$$
\begin{equation}
\begin{split}
I(X;Y) & =H(Y)-H(Y|X) \\
& = H(q(1-\epsilon)) - qH(1-\epsilon,\epsilon) \\
& = - q(1-\epsilon)\log{q(1-\epsilon)} - q(- ( (1-\epsilon)\log{(1-\epsilon)} + (\epsilon)\log{\epsilon})) \\
& = -q ( (1-\epsilon)\log{q(1-\epsilon)} - ( (1-\epsilon)\log{(1-\epsilon)} + (\epsilon)\log{\epsilon} ) ) \\
& = -q ( (1-\epsilon)\log{q(1-\epsilon)} - (1-\epsilon)\log{(1-\epsilon)} - (\epsilon)\log{\epsilon} ) \\
& = q(1-\epsilon)\log{\frac{1}{q(1-\epsilon)}} - q(1-\epsilon)\log{\frac{1}{(1-\epsilon)}} - q(\epsilon)\log{\frac{1}{\epsilon}} )
\end{split}
\end{equation}

$$\frac{d}{dq} = \frac{(\epsilon-1)(-\log{\frac{1}{q-\epsilon q}}) + \log{\frac{1}{1-\epsilon}} +1}{\log{2}}$$

I used wolfram alpha to find the root for the variable q:
$$q = -\frac{1-\epsilon}{\exp(\epsilon - 1)}$$
Which I believe will maximize the mutual information $I(X;Y)$ and as such achieve capacity.

\subsection*{(b) Consider a cascade of two channels, as shown below.  The output of the first channel directly input to the second.  Give the transition probabilities for the effective XZ channel.}

\begin{tabular}{c || c | c |}
  & z = 0 & z = 1 \\
  \hline
  x = 0 & $1-\epsilon+\epsilon \delta$ & $\epsilon(1-\delta)$ \\
  x = 1 & $\delta$ & $(1-\delta)$ \\
  \hline
\end{tabular}

\subsection*{(c) Find the value of $\delta$ (in terms of $\epsilon$) that makes the effective XZ channel symmetric, and find the corresponding capacity of the XZ channel.}

In order for the $XZ$ channel to be symmetric the following two equations must be true.
$$\delta=\epsilon (1 - \delta)$$
$$1-\epsilon + \delta \epsilon = 1-\delta$$

By solving this system of equations, we get:
$$\delta = \frac{\epsilon}{\epsilon+1}$$
We know that for a symmetric channel we have:
$$C = \log{|Z|} - H(row of transition matrix)$$
$$C = \log{2} - H(\frac{\epsilon}{\epsilon +1}, 1-\frac{\epsilon}{\epsilon+1})$$
Which is a Bernoulli function where $\alpha = \frac{\epsilon}{\epsilon +1}$:
$$C=1-\frac{\epsilon}{\epsilon +1}\log{\frac{\epsilon}{\epsilon +1} - (1-\frac{\epsilon}{\epsilon +1})\log{1-\frac{\epsilon}{\epsilon +1}}}$$

\subsection*{(d) What is the relationship between the capacity computed in (c) and $C_Z$? That is, explain the conditions where the capacity computed in (c) is less than, equal to, or greater than $C_Z$.}

We know from the data processing inequality, we know that we can never gain information through processing.  Therefore $\max{I(X;Y)} \geq \max{I(X;Z)}$  Equality occurs iff we can fully retrieve the value of X given the value of Y.  In this particular example, we could only accomplish this if $\epsilon=0$ since for all other cases, we can't fully determine any X given Y due to the transition probability when $X=0$.  For all other cases, the capacity computed in (c) will be less than $C_Z$.  The capacity for (c) will never be greater than $C_Z$ since the additional channel will increase the uncertainty.

\subsection*{(e) Suppose now that the output sequence from the first channel can be arbitrarily processed and the processed result can be sent on from Y to Z.  Find the resulting capacity of the combined system from $X$ to $Z$, $C_{XZ}$ (Hint: try to express $C_{XZ}$ in terms of $C_Z$ and $C_{YZ}$).}

$X, Y, and Z$ clearly form a Markov chain: $X \rightarrow Y \rightarrow Z$
We know from the data processing inequality that $I(X;Y)\geq I(X;Z)$
We know that the maximum effective capacity two channels in series is the the product of the two channels capacities.  That is:
$$C_{XZ} \leq C_{Z}*C_{YZ}$$
Since we process the the output sequence at Y we know that we either retain or lose mutual information, but we can never gain mutual information.  This means that the total capacity $C=\max{I(X;Z)}$ will decrease due to processing unless the processing results in a Y where $X = g(Y)$ in which case it will be equal.



\section{Problem 5}
\subsection*{Let the input random variable $X$ to a channel be uniformly distributed over the interval $-1/2 \leq x \leq +1/2$.  Let the output of the channel be $Y = X + Z$, where the noise random variable is uniformly distributed over the interval $-a/2 \leq z \leq +a/2$.}
The density of Y is as follows:\\
For $a < 1$:
\[ p_Y(y) =
  \begin{cases} 
      (\frac{1}{2a})(y+\frac{1+a}{2}) & -\frac{1+a}{2} \leq y \leq -\frac{1-a}{2} \\
      1 & -\frac{1-a}{2} \leq y \leq +\frac{1-a}{2} \\
      (\frac{1}{2a})(-y-\frac{1+a}{2}) & +\frac{1-a}{2} \leq y \leq +\frac{1+a}{2} \\
   \end{cases}
\]
For $a = 1$: Y is triangular over [-1, +1]\\
For $a>1$:
\[ p_Y(y) =
  \begin{cases} 
      y+\frac{1+a}{2} & -\frac{1+a}{2} \leq y \leq -\frac{1-a}{2} \\
      \frac{1}{a} & -\frac{1-a}{2} \leq y \leq +\frac{1-a}{2} \\
      -y-\frac{1+a}{2} & +\frac{1-a}{2} \leq y \leq +\frac{1+a}{2} \\
   \end{cases}
\]

\subsubsection{Find $I(X;Y)$ as a function of $a$.}
If $a<1$ the output Y will be conditionally uniformly distributed with probability $1-a$ over the interval $[-\frac{1-a}{2}, +\frac{1-a}{2}]$.  Y will have a triangular density with the base of the triangle having width a with probability a.
$$h(Y) = H(a) + (1-a)\ln(1-a)+a(\frac{1}{2}+\ln{a})$$
$$h(Y) = -a\ln{a} - (1-a)\ln{(1-a)} + (1-a)\ln{(1-a)} + \frac{a}{2} + a\ln{a}$$
$$h(Y) = \frac{a}{2}nats$$
If $a>1$ Y's trapezoidal density be scaled by a factor a:
$$h(Y)=\ln{a}+\frac{1}{2a}$$
Therefore:
\[ I(X;Y) =
  \begin{cases} 
	\frac{a}{2} - \ln{a} & a \leq 1 \\
	\frac{1}{2a} & a \geq 0 \\
   \end{cases}
\]

\subsubsection{For $a=1$ find the capacity of the channel when the input $X$ is peak-limited; that is, the range of $X$ is limited to $-1/2 \leq x \leq +1/2$.  What probability distribution on $X$ maximizes the mutual information $I(X;Y)$?}

$$I(X;Y) = h(Y) - h(Y|X) = h(Y) - h(Z)$$
Both X and Y are restricted to the interval $[-\frac{1}{2}, +\frac{1}{2}]$.  This means that since $Y = X + Z$, it is limited to the interval $[-1, +1]$.  The differential entropy of Y is $\leq$ the entropy of a uniformly distributed random variable on the interval $[-1, +1]$.  This means, $h(Y) \leq 1$.  This equation for maximum entropy achieves equality iff $x = \pm 1$ with probabilities of $\frac{1}{2}$ each.
$$I(X;Y) = h(Y) - h(Z) = 1 - 0 = 1$$
\[ \hat{X} =
  \begin{cases} 
	-\frac{1}{2} & y < 0 \\
	+\frac{1}{2} & y \geq 0 \\
   \end{cases}
\]













\end{document}  