\documentclass[11pt, oneside]{book}   	% use "amsart" instead of "article" for AMSLaTeX format
\usepackage[margin=1in]{geometry}                		% See geometry.pdf to learn the layout options. There are lots.
\geometry{letterpaper}                   		% ... or a4paper or a5paper or ... 
%\geometry{landscape}                		% Activate for for rotated page geometry
%\usepackage[parfill]{parskip}    		% Activate to begin paragraphs with an empty line rather than an indent
\usepackage{graphicx}				% Use pdf, png, jpg, or eps§ with pdflatex; use eps in DVI mode
\graphicspath{ {images/} }							% TeX will automatically convert eps --> pdf in pdflatex		
\usepackage{amssymb}
\usepackage{amsmath}

\setlength{\parindent}{0em}

\title{EECS 769 Homework 2}
\author{Elise McEllhiney}
\date{Sept. 20, 2016}							% Activate to display a given date or no date

\begin{document}
\maketitle

\section{The Value of a Question}
\subsection{Find the decrease in uncertainty $H(X)-H(X|Y)$}

\begin{equation}
\begin{split}
 H(X)-H(X|Y) & = H(Y)-H(Y|X)\\
 & = H(Y)-0\\
 & =H(Y)\\
 &= H(\alpha)
\end{split}
\end{equation}
Apparently any set $S$ with a given $\alpha$ is as good as any other.

\section{Random Questions}
\subsection{Show $I(X;Q,A)=H(A|Q)$. Interpret.}

\begin{equation}
\begin{split}
 I(X;Q,A) & = H(Q,A) - H(Q,A | X)\\
 & = H(Q) + H(A|Q) - H(Q|X) - H(A|Q,X)\\
 & \overset{(a)}{=} H(Q) + H(A|Q) - H(Q) - H(A|Q,X) \\
 & \overset{(b)}{=} H(Q) + H(A|Q) - H(Q) \\
 & = H(A|Q)
\end{split}
\end{equation}
(a) Since Q and X are independent $H(Q|X)=H(Q)$\\
(b) We know A given $(Q,X)$ so $H(A|Q,X)=0$\\\\
\textbf{Interpretation:} The uncertainty in the Answer given the Question, $H(A|Q)$, is the same as the uncertainty in $X$ removed by $(Q,A)$

\subsection{Now suppose that two i.i.d. questions $Q_1, Q_2 ~ r(q)$ are asked, eliciting answers $A_1$ and $A_2$.  Show that two questions are less valuable than twice a single question in the sense that $I(X;Q_1,A_1,Q_2,A_2) \leq 2I(X;Q_1,A_1)$.}

\begin{equation}
\begin{split}
 I(X;Q_1,A_1,Q_2,A_2) & = I(X;Q_1) + I(X;A_1|Q_1) + I(X;Q_2|Q_1,A_1) + I(X;A_2|Q_1, A_1, Q_2)\\
 & \overset{(a)}{=} I(X;A_1|Q_1) + I(X;Q_2|Q_1,A_1) + I(X;A_2|Q_1, A_1, Q_2)\\
 & = I(X;A_1|Q_1) + H(Q_2|A_1,Q_1) - H(Q_2|X,A_1,Q_1) + I(X;A_2|Q_1, A_1, Q_2)\\
 & \overset{(b)}{=} I(X;A_1|Q_1) + H(Q_2) - H(Q_2) + I(X;A_2|Q_1, A_1, Q_2)\\
 & = I(X;A_1|Q_1) + I(X;A_2|Q_1, A_1, Q_2)\\
 & = I(X;A_1|Q_1) + H(A_2|Q_1,A_1,Q_2) - H(A_2|X,Q_1,A_1,Q_2)\\
 & \overset{(c)}{=} (X;A_1|Q_1) + H(A_2|Q_1,A_1,Q_2)\\
 & \overset{(d)}{\leq} (X;A_1|Q_1) + H(A_2|Q_2)\\
 & \overset{(e)}{=} I(X;A_1|Q_1)+I(X;A_2|Q_2)\\
 & = 2I(X;A_1|Q_1)
\end{split}
\end{equation}
(a) $X$ and $Q_1$ are independent thus $I(X;Q_1)=0$\\
(b) $Q_2$ is independent of $Q_1$, $A_1$, and $X$\\
(c) Given $X$ and $Q_2$ we can completely determine $A_2$ so $H(A_2|X,Q_1,A_1,Q_2) = 0$\\
(d) Entropy never increases with conditioning
(e) Proven in 0.2.1

\section{Entropy of a disjoint mixture}
\subsection{Find $H(X)$ in terms of $H(X_1)$ and $H(X_2)$ and $\alpha$}

\[
    X= 
\begin{cases}
    X_1, & \text{with probability } \alpha\\
    X_2, & \text{with probability } 1-\alpha
\end{cases}
\]
\[
    f(X)= 
\begin{cases}
    0, & \text{when } X=X_1\\
    1, & \text{when } X=X_2
\end{cases}
\]
\begin{equation}
\begin{split}
 H(X) & = H(X|f(X))\\
 & = H(f(X) + H(X|f(X))\\
 & = H(f(X)) + p(f(X)=0)H(X|f(X)=0) +p(f(X)=1)H(X|f(X)=1)\\
 & = H(\alpha) + \alpha H(X_1) + (1-\alpha)H(X_2)\\
\end{split}
\end{equation}
where $H(\alpha) = -\alpha log( \alpha) - (1-\alpha)log(1-\alpha)$

\subsection{Maximize over $\alpha$ to show that $2^{H(X)} \leq 2^{H(X_1)}+2^{H(X_2)}$ and interpret using the notion that $2^{H(X)}$ is the effective alphabet size.}

$$\frac{dH(X))}{d\alpha} = H(X_1)-H(X_2)+log(\frac{1-\alpha}{\alpha})$$
Set the derivative to 0
$$0 = H(X_1)-H(X_2)+log(\frac{1-\alpha}{\alpha})$$
$$-H(X_1)+H(X_2) = log(\frac{1-\alpha}{\alpha})$$
$$2^{-H(X_1)+H(X_2)} = \frac{1-\alpha}{\alpha}$$
$$\alpha = \frac{1}{2^{-H(X_1)+H(X_2)} + 1}$$
$$\alpha = \frac{2^{H(X_1)}}{2^{H(X_1)} + 2^{H(X_2)}}$$
\begin{equation}
\begin{split}
H(X) & = H(\alpha) + \alpha H(X_1) + (1-\alpha)H(X_2)\\
& \leq log(2^{H(X_1)} + 2^{H(X_2)})
\end{split}
\end{equation}
$$2^{H(X)} \leq 2^{H(X_1)} + 2^{H(X_2)}$$
\textbf{Interpretation:} The effective alphabet size of X, a combination of $X_1$ and $X_2$, is less than or equal to their effective alphabet sizes
\subsection{Let $X_1$ and $X_2$ be uniformly distributed over their alphabets.  What is the maximizing $\alpha$ and the associated $H(X)$}
The maximizing $\alpha$ is worked out above.
$$H(X) = log(2^{H(X_1)} + 2^{H(X_2)})$$
The effective alphabet size of two disjoint uniformly distributed random variables is equal to the sum of their effective alphabet sizes

\section{Bottleneck}
\subsection{Show that the dependence of $X_1$ and $X_3$ is limited by the bottleneck by proving that $I(X_1;X_3) \leq log(k)$}

\begin{equation}
\begin{split}
I(X_1;X_3) & \overset{(a)}{\leq} I(X_1;X_2)\\
& = H(X_2) - H(X_2|X_1)\\
& \leq H(X_2)\\
& \leq log(k)
\end{split}
\end{equation}
(a) Since we are given that we are working with a Markov chain, we know that mutual information between $X_1$ and $X_3$ is less than the mutual information between $X_1$ and $X_2$

\subsection{Evaluate $I(X_1;X_3)$ for $k=1$, and conclude that no dependence can survive such a bottleneck}

\begin{equation}
\begin{split}
I(X_1;X_3) & \leq log(k)\\
& \leq log(1) = 0
\end{split}
\end{equation}
Since entropy is always nonnegative we know that $I(X_1;X_3)=0$ and thus for $k=1$ we know that $X_1$ and $X_3$ must be independent

\section{Relative entropy is not symmetric}
\subsection{Calculate $H(p), H(q), D(p||q)$ and $D(q||p)$}

$$H(p) = -\sum p(x)log(p(x))=\frac{1}{2}log(2)+\frac{1}{4}log(4)+\frac{1}{4}log(4) = 1.5 bits$$
$$H(q) = -\sum q(x)log(q(x))=\frac{1}{3}log(3)+\frac{1}{3}log(3)+\frac{1}{3}log(3) = 1.585 bits$$
$$D(p||q) = \sum p(x)log\frac{p(x)}{q(x)}=\frac{1}{2}log(\frac{3}{2})+\frac{1}{4}log(\frac{3}{4})+\frac{1}{4}log(\frac{3}{4})=0.085$$
$$D(q||p) = \sum q(x)log\frac{q(x)}{p(x)}=\frac{1}{3}log(\frac{2}{3})+\frac{1}{3}log(\frac{4}{3})+\frac{1}{3}log(\frac{4}{3})=0.082$$

\subsection{Verify that in this case $D(p||q) \neq D(q||p)$}

As we can see from the above equations $D(p||q) \neq D(q||p)$ since $0.85 \neq 0.082$

\section{Conditional mutual information}
\subsection{Find the mutual information}
Since we know that we must have an even number of ones, we know that the $n^{th}$ digit is determined by the previous $n-1$ digits.  Since we know that any sequence with an even number of digits is equally likely, we know that any $n-1$ or fewer of the digits are independent.  
\\For $k \leq n-1$ we know that:
$$I(X_{k-1};X_k|X_1, X_2,...,X_{k-2}) = 0$$
However, if we know $n-1$ digits, the last digit is deterministic.
$$I(X_{n-1};X_n|X_1, X_2, ..., X_{n-2}) = H(X_n|X_1,X_2,...,X_{n-2}) - H(X_n|X_1,X_2,...,X_{n-1})=1bit$$

\section{Conditional mutual information vs unconditional mutual information}
\subsection{Give examples of joint random variables such that $X, Y,$ and $Z$ such that:}
\subsubsection{(a)$(I(X;Y|Z) < I(X;Y)$}
X is a uniform binary random variable where $Y=X$ and $Z=Y$
$$I(X;Y) = H(X) - H(X|Y) = H(X) = 1$$
$$I(X;Y|Z) = H(X|Z)-H(X|Y,Z) = 0$$
\subsubsection{(b)$I(X;Y|Z) > I(X;Y)$}
X and Y are independent uniform binary random variables.  $Z = X+Y$
$$I(X;Y)= H(X) - H(X|Y) = 0$$
$$I(X;Y|Z)=H(X|Z)-H(X|Y,Z) = H(X|Z) = 1/2$$

\section{Entropy of initial conditions}
\subsection{Prove that $H(X_0|X_n)$ is non-decreasing with $n$ for any Markov chain}
The data processing inequality states that for any Markov chain, $I(X;Y) \geq I(X;Z)$
$$I(X_0;X_{n-1}) \geq I(X_0;X_n)$$
$$H(X_0)-H(X_0|X_{n-1} \geq H(X_0)-H(X_0|X_n)$$
Thus, $H(X_0|X_n)$ is non-decreasing as $n$ increases for any Markov chain

\section{A metric}
$$\rho(x,y) \geq 0$$
$$\rho(x,y) = \rho(y,x)$$
$$\rho(x,y) = 0 iff x=y$$
$$\rho(x,y)+\rho(y,z) \geq \rho(x,z)$$
\subsection{Show that $\rho (X,Y) = H(X|Y) + H(Y|X)$ satisfies the first, second and fourth properties above.  If we say that $X=Y$ if there is a one-to-one function mapping $X$ to $Y$, then the third property is also satisfied, and $\rho (X,Y)$ is a metric}
$$\rho(X,Y)=H(X|Y)+H(Y|X)$$

1) $\rho(X,Y) \geq 0$ since entropy is always nonnegative.\\
2) $\rho(X,Y)=\rho(Y,X)$ since $H(X|Y)+H(Y|X) = H(Y|X)+H(X|Y)$\\
3) We know that $H(X|Y)=0$ iff $X=g(Y)$ and that $H(Y|X)=0$ iff $Y=g(X)$.  From this we know that $\rho(X,Y)=0$ iff $Y=g(X)$ and $X=g(Y)$.  Thus, we know that $p(X,Y)=0$ iff there is a one-to-one mapping from $X$ to $Y$\\
4)$\rho(x,y)+\rho(y,z) \geq \rho(x,z)$ since:
\begin{equation}
\begin{split}
H(X|Y) + H(Y|Z) & \geq H(X|Y,Z) + H(Y|Z)\\
& = H(X,Y|Z)\\
& = H(X|Z) + H(Y|X,Z)\\
& \geq H(X|Z)
\end{split}
\end{equation}

\subsection{Verify that $\rho (X,Y)$ can also be expressed as}
\begin{equation}\label{}
\begin{split}
 \rho(X,Y) & = H(X) + H(Y) - 2I(X;Y)\\
 & = H(X,Y) - I(X;Y)\\
 & = 2H(X,Y) - H(X) - H(Y)
\end{split}
\end{equation}\\
$$H(X|Y) = H(X) - I(X;Y)$$
$$H(Y|X) = H(Y) - I(X;Y)$$
Therefore
$$\rho(X,Y) = H(X) - I(X;Y) + H(Y) - I(X;Y) = H(X) + H(Y) - 2I(X;Y)$$
$$H(X,Y) = H(X)+H(Y) - I(X,Y)$$
Thus, combining the previous two equations, we get:
$$\rho(X,Y)=H(X,Y)-I(X,Y)$$
$$I(X;Y)=H(X)+H(Y)-H(X,Y)$$
Combining the previous two equations we get:
$$\rho(X,Y)=2H(X,Y)-H(X)-H(Y)$$

\section{Run length coding}
\subsection{Show all equalities and inequalities, and give simple bounds on all differences}
We know that $H(R) \leq H(X)$ since we know that the run lengths, $R$, are a function of $X_1, X_2,...,X_n$\\
Knowing both the run lengths and any $X_i$ in the sequence, we can determine the entire sequence $X_1, X_2,...,X_n$
\begin{equation}\label{}
\begin{split}
 H(X_1, X_2,...,X_n) & = H(X_i,R)\\
 & = H(R) + H(X_i|R)\\
 & \leq H(R) + H(X_i)\\
 & \leq H(R) + log(2)\\
 & = H(R) + 1
\end{split}
\end{equation}

\section{Pure randomness and bent coins}
\subsection{Give reasons for the following inequalities}
\begin{equation}
\begin{split}
 nH(p) & \overset{(a)}{=} H(X_1,...,X_n)\\
 & \overset{(b)}{\geq} H(Z_1, Z_2,...,Z_K,K)\\
 & \overset{(c)}{=} H(K) + H(Z_1,...,Z_K|K)\\
 & \overset{(d)}{=} H(K) + E(K)\\
 & \overset{(e)}{\geq} EK
\end{split}
\end{equation}
(a) $X_i=1$ with the probability $p$.  Also, we know that $X_1, X_2,...,X_n$ are independent and identically distributed.  Thus, we can conclude that $nH(p) = H(X_1,X_2,...,X_n)$.  In other words, the entropy of a sequence is the product of the entropies of the random variables\\
(b) We know that $(Z_1,Z_2,...,Z_K) = f(X_1, X_2,...,X_n)$.  Since we know that the entropy of a random variable is always greater than or equal to the entropy of a function of that random variable, we know that: $H(X_1,X_2,...,X_n) \geq H(Z_1,Z_2,...,Z_K)$.  Since K is a function of $Z_1,Z_2,...,Z_K$ we know that $H(K|Z_1,Z_2,...,Z_K) = 0$.  Thus:
$$H(Z_1,Z_2,...,Z_K,K) = H(Z_1,Z_2,...,Z_K) + H(K|Z_1,Z_2,...,Z_K) = H(Z_1,Z_2,...,Z_K)$$
(c)Chain rule\\
(d)Since $Z_1,Z_2,...,Z_K$ are random bits with $H(Z_i)=1bit$.\\
(e)Entropy is non-negative\\
Thus no more than $nH(p)$ fair coin tosses can be derived from $(X_1,...,X_n)$, on the average.
\subsection{Exhibit a good map f on sequences of length 4}
Since it is an unfair coin and we don't know $p$ there are not very many ways to simulate a sequence of fair coin-flips.  However, even without knowing $p$ we know that the flips are i.i.d. so we know that a series of 4 flips having a given number of heads occurring is equally likely.
\begin{center}
\begin{tabular}{ c | c | c | c | c | c |}
0000 $\rightarrow$ NULL & & & & & \\
0001 $\rightarrow$ 00 & 0010 $\rightarrow$ 01 & 0100 $\rightarrow$ 10 & 1000 $\rightarrow$ 11 & & \\
0011 $\rightarrow$ 00 & 0101 $\rightarrow$ 01 & 0110 $\rightarrow$ 10 & 1010 $\rightarrow$ 11 & 1001 $\rightarrow$ 0 & 1100 $\rightarrow$ 1 \\
0111 $\rightarrow$ 00 & 1011 $\rightarrow$ 01 & 1101 $\rightarrow$ 10 & 0111 $\rightarrow$ 11 & & \\
1111 $\rightarrow$ NULL & & & & &
\end{tabular}
\end{center}
















\end{document}  