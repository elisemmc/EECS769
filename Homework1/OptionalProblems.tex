\documentclass[11pt, oneside]{book}   	% use "amsart" instead of "article" for AMSLaTeX format
\usepackage{geometry}                		% See geometry.pdf to learn the layout options. There are lots.
\geometry{letterpaper}                   		% ... or a4paper or a5paper or ... 
%\geometry{landscape}                		% Activate for for rotated page geometry
%\usepackage[parfill]{parskip}    		% Activate to begin paragraphs with an empty line rather than an indent
\usepackage{graphicx}				% Use pdf, png, jpg, or eps§ with pdflatex; use eps in DVI mode
\graphicspath{ {images/} }							% TeX will automatically convert eps --> pdf in pdflatex		
\usepackage{amssymb}

\setlength{\parindent}{0em}

\title{EECS 769 Optional Problems for Homework 1}
\author{Elise McEllhiney}
\date{Sept. 8, 2016}							% Activate to display a given date or no date

\begin{document}
\maketitle
\section{Three events $E_1$, $E_2$, and $E_3$ defined on the same space, have probabilities $P(E_1) = P(E_2) = P(E_3) = 1/4$.  Let $E_0$ be the event that one or more of the events $E_1$, $E_2$ and $E_3$ occurs.}
\subsection{Find $P(E_0)$ when:}
\subsubsection{The events $E_1$, $E_2$, and $E_3$ are disjoint.}
$$P(E_0)=P(E_1)+P(E_2)+P(E_3)=1/4 + 1/4+ 1/4 = 3/4$$
The probability of one of the events occurring is equal to the sum of the individual probabilities of the disjoint events.
\subsubsection{The events $E_1$, $E_2$, and $E_3$ are statistically independent.}
$$P(E_0) = 1 - (P(E_1=0)*P(E_2=0)*P(E_3=0)) = 1-(3/4)*(3/4)*(3/4) = 1-(27/64) = 37/64$$
If they are independent, the outcome of one event doesn't give any information on another event.  This means that the probability that one or more events will occur is $1-P(no events occur)$
\subsubsection{The events $E_1$, $E_2$, and $E_3$ are in fact names for the same event.}
$$P(E_0) = 1/4$$
Since they are the same event, we know $P(E_0)=P(E_1)$ since if one event occurs, all events occur, and one event doesn't occur, no events occur.  The events are completely dependent on each other.
\subsection{Find the maximum value $P(E_0)$ can assume when:}
\subsubsection{Nothing is known about the independent or disjointness of $E_1$, $E_2$ and $E_3$.}
If we know nothing about their independence or disjointedness, the maximum value that $P(E_0)$ can assume is 1.  If $P(E_3)$ is always true when $P(E_1)$ and $P(E_2)$ are false then we could be assured that we would always have at least one of them be true.  Each individual event could maintain their original probabilities of a single event occurring, but the combination could guarantee that at least one of them would always be true. 
\subsubsection{Is is known that $E_1$, $E_2$ and $E_3$ are pairwise independent.  i.e., that the probability of realizing both $E_i$ and $E_j$ is $P(E_i)P(E_j), 1 \leq i \neq j \leq 3$ but nothing is known about the probability of realizing all three events together.}
Again, $P(E_0)$ could feasibly take on the quantity one if $P(E_3)$ was fully dependent on the vector $(E_1,E_2)$.  It might be independent of the two outcomes individually, and still a function of the vector formed by the two values.

\section{A dishonest gambler has a loaded die which turns up the number 1 with probability 2/3 and the numbers 2 to 6 with the probability 1/15 each.  Unfortunately, he has left his loaded die in a box with two honest dice and cannot tell them apart.}
\subsection{He picks up one die (at random) from the box, rolls is once, and the number 1 appears.  Conditional on this result, what is the probability that he picked up the loaded die?}
$$P(A|B)=\frac{P(A \cap B)}{P(B)}$$
$$P(loaded|1)=\frac{P(loaded)*\frac{2}{3}}{P(loaded)*\frac{2}{3}+P(fair)*\frac{1}{6}}=\frac{\frac{1}{3}*\frac{2}{3}}{\frac{1}{3}*\frac{2}{3}+\frac{2}{3}*\frac{1}{6}}=\frac{2}{3}$$
We know that if his choice of die was random the probability of him picking the loaded die is $\frac{1}{3}$ and the probability of the loaded die rolling a 1 is $\frac{2}{3}$ so we can use the property of conditional probability listed above to compute the probability of him picking the loaded die given that it rolled a one on the first try.
\subsection{He now rolls the dice once more and it comes up 1 again.  What is the possibility after this second rolling that he has picked up the loaded die?}
$$P(A|B)=\frac{P(A \cap B)}{P(B)}$$
$$P(loaded|1,1)=\frac{P(loaded)*(\frac{2}{3})^2}{P(loaded)*(\frac{2}{3})^2+P(fair)*(\frac{1}{6})^2}=\frac{\frac{1}{3}*(\frac{2}{3})^2}{\frac{1}{3}*(\frac{2}{3})^2+\frac{2}{3}*(\frac{1}{6})^2}=\frac{8}{9}$$
Again, we know the probability of either the unfair die or the fair die rolling two ones in a row.  We just plug that into the aforementioned equation to work out the probability that we have the loaded die given that we rolled two ones in a row.
\end{document}