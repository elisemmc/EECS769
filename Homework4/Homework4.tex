\documentclass[11pt, oneside]{book}   	% use "amsart" instead of "article" for AMSLaTeX format
\usepackage[margin=1in]{geometry}                		% See geometry.pdf to learn the layout options. There are lots.
\geometry{letterpaper}                   		% ... or a4paper or a5paper or ... 
%\geometry{landscape}                		% Activate for for rotated page geometry
%\usepackage[parfill]{parskip}    		% Activate to begin paragraphs with an empty line rather than an indent
\usepackage{graphicx}				% Use pdf, png, jpg, or eps§ with pdflatex; use eps in DVI mode
\graphicspath{ {images/} }							% TeX will automatically convert eps --> pdf in pdflatex
\usepackage{amssymb}
\usepackage{amsmath}

\setlength{\parindent}{0em}

\title{EECS 769 Homework 4}
\author{Elise McEllhiney}
\date{Nov. 3, 2016}							% Activate to display a given date or no date

\begin{document}
\maketitle

\section{Channel with Jammer}
\subsection*{A sender transmits a random variable $X$ through a channel with output $Y$ given by
$$Y=X+Z \mod 2$$
where $Z$ is the signal inserted by a jammer.  X and Z are independent.  X and Z can only take binary values 0 or 1.  The sender is limited in energy so that $E_X[X^2] \leq 1/2$ and the jammer is limited in energy so that $E_Z[Z^2] \leq 1/4$.  The sender seeks to maximize mutual information $I(X;Y)$, while the jammer seeks to minimize it.  What is the policy followed by the sender, by the jammer, and the resulting mutual information.}
Since Z is independent of X, we know that:
$$C=\max_{p(x)}I(X;Y)=\max_{p(x)}H(X)-H(X|Y)=1-(2-\frac{3}{4}\log{3})=\frac{3}{4}\log{3}-1$$
Note: $H(X|Y)=-\sum_{x\in \mathcal{X}}\sum_{y\in \mathcal{Y}} \frac{3}{8}\log{\frac{3}{4}} + \frac{1}{8}\log{\frac{1}{4}} + \frac{1}{8}\log{\frac{1}{4}} + \frac{1}{8}\log{\frac{1}{4}} = 2-\frac{3}{4}\log{3}$\\
The sender will maximize mutual information by sending X as a uniform distribution with $p(X=0)=1/2$ and $p(X=1)=1/2$.
The jammer would maximize disruption by sending Z as a uniform distribution.  However, since it can send a maximum of 1 out of every 4 bits, it should do so.  so it sends Z with the distribution $p(Z=0)=3/4$ and $p(Z=1)=1/4$

\section{Statistician}
\subsection*{One is given a communication channel with transition probabilities $p(x|y)$ and channel capacity $C = \max_{P_X} I(X;Y)$.  A helpful statistician preprocesses the output by forming $\hat{Y}=g(Y)$.  He claims that this will strictly improve the capacity.}
\subsubsection{(a) Show that he is wrong.}
We know that he is wrong by the data processing inequality.  $X \rightarrow Y \rightarrow \hat{Y}$ is a Markov chain.  And as such, by the data processing inequality we know that:
$$I(X;Y) \geq I(X;\hat{Y})$$
If $\hat{p}(x)$ is the maximizing distribution, we then know:
$$C=max_{p(x)}I(X;Y) \geq I(X;Y)_{p(x)=\hat{p}(x)} \geq I(X;\hat{Y})_{p(x)=\hat{p}(x)}=max_{p(x)}I(X;\hat{Y})$$
\subsubsection{(b) Under what conditions does he not strictly decrease the capacity?}
He doesn't strictly decrease capacity if we have equality in the data processing inequality.  In other words, he doesn't lose capacity if $X \rightarrow \hat{Y} \rightarrow Y$ also forms a Markov chain.

\section{Channel with variable noise}
\subsection*{Find the channel capacity of the following discrete memoryless channel:
$$ Y=X+Z \quad X \in \{0,1\} \quad Z \in \{0,a\} $$
where $Pr\{Z=0\}=Pr\{Z=a\}=1/2$ and $a$ is a non-zero integer.  Assume that $Z$ is independent of $X$.\\Note that the channel capacity depends on the value of $a$.}
If $a = 0$ : We know that $Y = X$.  Therefore $\max I(X;Y)=\max H(X) = 1 bits/transmission$\\\\
If $a = 1$ : We know $Y=\{0,1,2\}$.  This is a binary erasure with $a=1/2$ since $Pr\{Z=a\}=1/2$.  Therefore the capacity when $a = 1$ is $1-a = 1-1/2 = \frac{1}{2} bits/transmission$\\\\
If $a=-1$ : We know $Y=\{-1,0,1\}$.  Again, this is a binary erasure channel with $a=1/2$.  Therefore, capacity when $a=-1$ is $1-a=1-1/2=\frac{1}{2}bits/transmission$\\\\
If $a \neq -1,0,1$ : We know $Y=\{0, 1, a, 1+a\}$.  Now we can see that if we know $Y$, we will know $X$ and $H(X|Y)=0$.  Therefore, $\max I(X;Y)=\max H(X) = 1bits/transmission$

\section{Noisy typewriter channel}
\subsection*{Consider the discrete memoryless channel
$$Y=X+Z(\mod{11})$$
where $Pr\{Z=1\}=Pr\{Z=2\}=Pr\{Z=3\}=1/3$ and $X \in {0,1,...,10}$.  Assume that $Z$ is independent of $X$.}
\subsubsection{(a) Find the capacity}
Since Z is a uniform distribution and independent of X, we know that:
$$H(Z)=H(Z|X)=H(Y|X)=\log 3$$
$$C=\max_{p(x)}I(X;Y)=\max_{p(x)}H(Y)-H(Y|X)=\max_{p(x}H(Y)-\log 3= \log 11 - \log 3= \log \frac{11}{3} bits/transmission$$
\subsubsection{(b) What is the maximizing $p^*(x)$}
This is achieved when $Y$ has a uniform distribution.  Or, $p(X=i)=\frac{1}{11}$ for $i=\{0,1,...,10\}$

\section{Erasure channel}
\subsection*{Let $\{\mathcal{X}, p(x|y), \mathcal{Y}\}$ be a discrete memoryless channel with capacity $C$.  Suppose this channel is immediately cascaded with an erasure channel $\{ \mathcal{Y}, p(s|y), \mathcal{S} \}$ that erases $\alpha$ of its symbols.  Specifically, $S=\{y_1, y_2, ..., y_m, e\}$, and 
$$Pr\{S=y|X=x\}=\bar{\alpha}p(y|x), \quad y\in \mathcal{Y}$$
$$Pr\{S=e|X=x\}=\alpha$$
Determine the capacity of this channel}
\[ \begin{cases} 
      1 & Z = e \\
      0 & otherwise
   \end{cases}
\]
Now we have a variable independent of $X$ and we know $p(Z=1)=\alpha$ and $p(Z=0)=1-\alpha$
\begin{equation}\label{}
    \begin{split}
        I(X;S) & = H(S)-H(S|X)\\
        & = H(S,Z)-H(S,Z|X) + H(Z)+H(S|Z)-H(Z|X)-H(S|X,Z)\\
        & = I(X;Z)+I(S;X|Z)\\
        & = 0 + \alpha I(X;S|Z=1)+(1-\alpha)I(X;S|Z=0)
    \end{split}
\end{equation}
Since $Z=1$ means we encountered an erasure, ie $S=e$.  We know that $H(S|Z=1)=H(S|X,Z=1)=0$.  We also know that when $Z=0$, $S=Y$ and therefore $I(X;S|Z=0)=I(X;Y)$
$$I(X;S)=(1-\alpha)I(X;Y)$$  A channel cascaded with an erasure channel will have $(1-\alpha)$ times the capacity of the original channel (without the erasure channel).

\section{Source and channel}
\subsection*{We wish to encode a Bernoulli($\alpha$) process $V_1, V_2,...$ for transmission over a binary symmetric channel with crossover probability $p$.  Find the conditions on $\alpha$ and $p$ so that the probability of error $P(\hat{V^n} \neq V^n)$ can be made to go to zero as $n \rightarrow \infty$}

By the source-channel theorem we know that to have a probability of error approach 0 as $n \rightarrow \infty$ we need to have the entropy of the source be less than the channel capacity.
$$H(\alpha) + H(p) < 1 \quad \quad \quad \alpha^{\alpha}(1-\alpha)^{1-\alpha}p^p(1-p)^{1-p} < \frac{1}{2}$$

\section{Capacity}
\subsection*{Find the capacity of}
(Used in part b) Like the effective alphabet size of two combined alphabets, we'll find that $2^C=2^{C_1}+2^{C_2}$ when we have two channels in parallel.
$$X=
\begin{cases}
X_1 \quad \alpha\\
X_2 \quad 1-\alpha
\end{cases}$$
$$Z=
\begin{cases}
1 \quad X_1\\
2 \quad X_2
\end{cases}$$
Thus, since we have disjoint channels we know that 
$$X \rightarrow Y \rightarrow Z$$
$$I(X;Y,Z)=I(X|Z)+I(X;Y|Z)=I(X;Y)+I(X;Z|Y)$$
Since elements in a Markov chain are conditionally independent given earlier variables in the chain: $I(X;Z|Y)=0$
$$I(X;Y,Z)=H(\alpha)+\alpha I(X_1;Y_1)+(1-\alpha)I(X_2;Y_2)$$
%Maximizing on $\alpha$ we get: $\alpha = \frac{2^{C_1}}{2^{C_1}+2^{C_2}}$
$$C=\log(2^{C_1}+2^{C_2})$$
%The capacity of two disjoint channels: $C=\max_{\alpha}H(\alpha}C_1+(1-\alpha)C_2$
\subsubsection{(a) Two parallel BSCs}
We have a symmetric channel.  We know the highest capacity is achieved by a uniform distribution.
$$C=\max_{p(x)}I(X;Y)=\max_{p(x)}H(Y)-H(Y|X)$$
$$\leq \log{|Y|}-H(p) = 2-H(p) $$
\subsubsection{(b) BSC and a single symbol}
I was unable to do this particular one without using the rule I found for parallel channels mentioned above 
$$C_1=1-H(p) \quad \quad \quad C_2=0$$
$$2^C = 2^{C_1}+1$$
$$C=log(2^{1-H(p)}+1)$$
\subsubsection{(c) BSC and a ternary channel}
Since we have a weekly symmetric channel (pg 191).
$$C=\log{|\mathcal{Y}|}-H(\text{row of transition matrix})=\log{5}-(\frac{1}{2}\log{\frac{1}{2}}+\frac{1}{2}\log{\frac{1}{2}})=\log{5}-\log{2}=\log{\frac{5}{2}}$$
We can also solve it the following way:\\
Ternary Channel : $$C_1=\max_{p(x)}I(X;Y)=\max{p(x)}H(Y)-H(Y|X) \leq \log{3} - H(1/2) = \log{3}-1$$
BSC: $$C_2=1-H(p)=0$$
In Parallel:$$2^C=\frac{3}{2}+1 \quad => \quad C = \log{\frac{5}{2}}$$
\subsubsection{(d) Ternary channel}
Since we have a weekly symmetric channel
$$C=\log{|\mathcal{Y}|}-H(\text{row of transition matrix})=\log{3}-H(\frac{1}{3}, \frac{2}{3})=\log{3}-(\log{3}-\frac{2}{3})=\frac{2}{3}$$
The capacity is achieved by a uniform distribution of the input alphabet.

























\end{document}  