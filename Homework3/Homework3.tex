\documentclass[11pt, oneside]{book}   	% use "amsart" instead of "article" for AMSLaTeX format
\usepackage[margin=1in]{geometry}                		% See geometry.pdf to learn the layout options. There are lots.
\geometry{letterpaper}                   		% ... or a4paper or a5paper or ... 
%\geometry{landscape}                		% Activate for for rotated page geometry
%\usepackage[parfill]{parskip}    		% Activate to begin paragraphs with an empty line rather than an indent
\usepackage{graphicx}				% Use pdf, png, jpg, or eps§ with pdflatex; use eps in DVI mode
\graphicspath{ {images/} }							% TeX will automatically convert eps --> pdf in pdflatex
\usepackage{amssymb}
\usepackage{amsmath}

\setlength{\parindent}{0em}

\title{EECS 769 Homework 3}
\author{Elise McEllhiney}
\date{Oct. 4, 2016}							% Activate to display a given date or no date

\begin{document}
\maketitle

\section{Data processing}
\subsection*{Let $X_1 \rightarrow X_2 \rightarrow X_3 ... \rightarrow X_n$ form a Markov chain in this order; i.e., let 
$$p(x_1,x_2,...,x_n)=p(x_1)p(x_2|x_1)...p(x_n|x_{n-1})$$ 
Reduce $I(X_1;X_2,...,X_n)$ to it's simplest form}

We can use the chain rule
$$I(X_1;X_2,...,X_n) = I(X_1;X_2)+I(X_1;X_3|X_2)+...+I(X_1;X_n|X_2,...,X_{n-2})$$
By the  we know that, for a Markov chain, $X_i$ and $X_{i+2}$ are conditionally independent given $X_{i+1}$ we know that everything in the above equation goes to zero except $I(X_1,X_2)$.  This can also be said as $I(X_1;X_i) \subseteq I(X_1;X_2)$ for $i \geq 2$
Hence,
$$I(X_1;X_2,...,X_n)=I(X_1,X_2)$$

\section{Markov's inequality for probabilities}
\subsection*{Let $p(x)$ be a probability mass function.  Prove, for all $d \geq 0$. $$Pr\{p(X) \geq d\} log (\frac{1}{d}) \geq H(X)$$}

% LOOK INTO FOLLOWING EQUATION, MAKE ADJUSTMENTS
\begin{equation}\label{}
\begin{split}
P(p(X)) < d) log(\frac{1}{d}) & = \sum_{x} p(x)log(\frac{1}{d}) \\
& \text{where $p(x) < d$}\\
& \leq \sum_{x} p(x)log(\frac{1}{p(x)}) \\
& \leq \sum_{x} p(x)log(\frac{1}{p(x)}) \\
& = H(X)  
\end{split}
\end{equation}\\

\section{Fano}
\subsection*{We are given the following joint distribution on $(X,Y)$}
\begin{center}
\begin{tabular}{| c  c | c c c |}
\hline
& Y & & & \\
X & & a & b & c \\
\hline
& 1 & $\frac{1}{6}$ & $\frac{1}{12}$ & $\frac{1}{12}$ \\
& 2 & $\frac{1}{12}$ & $\frac{1}{6}$ & $\frac{1}{12}$ \\
& 3 & $\frac{1}{12}$ & $\frac{1}{12}$ & $\frac{1}{6}$ \\
\hline
\end{tabular}
\end{center}

\subsection*{Let $\hat{X} (Y)$ be an estimator for $X$ (based on $Y$) and let $P_e = Pr \{ \hat{X}(Y) \neq X \}$}
\subsection{Find the minimum probability of error estimator $\hat{X} (Y)$ and the associated $P_e$}
$$\hat{X} (a) = 1$$
$$\hat{X} (b) = 2$$
$$\hat{X} (c) = 3$$
\begin{equation}\label{}
\begin{split}
\therefore P_e & = Pr(1,b) + Pr(1,c) + Pr(2,a) + Pr(2,c) +Pr(3,a) + Pr(3,b) \\
& = \frac{1}{12} + \frac{1}{12} + \frac{1}{12} + \frac{1}{12} + \frac{1}{12} + \frac{1}{12} \\
& = \frac{1}{2}
\end{split}
\end{equation}

\subsection{Evaluate Fano's inequality for this problem and compare}
$$H(P_e) + P_e log(|\mathcal{X}|-1) \geq H(X|Y)$$\\
$$H(\frac{1}{2}, \frac{1}{4}, \frac{1}{4}) = -\frac{1}{2}\log\frac{1}{2} - \frac{2}{4}\log\frac{1}{4} = \frac{1}{2} + 1 = 1.5 bits$$\\
$$H(P_e) = -\frac{6}{12} \log\frac{1}{12} = 1.79bits$$
\begin{equation}\label{}
\begin{split}
H(X|Y) & = H(X|Y=a)Pr(y=a) + H(X|Y=b)Pr(y=b) + H(X|Y=c)Pr(y=c) \\
& = H(\frac{1}{2}, \frac{1}{4}, \frac{1}{4})Pr(y=a) + H(\frac{1}{2}, \frac{1}{4}, \frac{1}{4})Pr(y=b) + H(\frac{1}{2}, \frac{1}{4}, \frac{1}{4})Pr(y=c) \\
& = H(\frac{1}{2}, \frac{1}{4}, \frac{1}{4})(Pr(y=a)+Pr(y=b)+Pr(y=c)) \\
& = H(\frac{1}{2}, \frac{1}{4}, \frac{1}{4}) \\
& = 1.5bits \\
\end{split}
\end{equation}
$$H(P_e) + P_e log(|\mathcal{X}|-1) \geq H(X|Y)$$
$$1.79bits + \frac{1}{2} log(3-1) \geq 1.5$$
$$1.79bits + \frac{1}{2} log(3-1) \geq 1.5$$
$$1.79 + 0.5  \geq 1.5$$
$$2.29 \geq 1.5$$
We can see that Fano's inequality holds true

\section{Fano's inequality}
\subsection*{Let $Pr(X=i) = p_i, i = 1, 2, ..., m$ and let $p_i \geq p_2 \geq p_3 \geq ... \geq p_m$.  The minimal probability of error predictor of $X$ is $\hat{X} = 1$, with resulting probability or error $P_e = 1-p_1$. Maximize $H(\bold{p})$ subject to the constraint $1-p_1 = P_e$ to find a bound on $P_e$ in terms of $H$.  This is Fano's inequality in the absence of conditioning}

\begin{equation}\label{}
\begin{split}
H(\bold{p}) & = -p_1 \log p_1 - \sum_{i=2}^{m} p_i \log p_i \\
& = -p_1 \log p_1 - \sum_{i=2}^{m} P_e \frac{p_i}{P_e} \log \frac{p_i}{P_e} - P_e \log P_e \\
& = H(P_e) + P_e H (\frac{p_2}{P_e}, \frac{p_3}{P_e},...,\frac{p_m}{P_e}) \\
& \leq H(P_e) + P_e \log (m-1)
\end{split}
\end{equation}
We know that maximum entropy is achieved through a uniform distribution.  Hence, 
$$H(X) \leq H(P_e) + P_e \log(m-1)$$
Since $H(P_e)$ is $\geq 1$
$$H(X) \leq 1 + P_e \log(m-1)$$
$$H(X)-1 \leq P_e \log (m-1)$$
$$\frac{H(X)-1}{\log(m-1)} \leq P_e$$

\section{Maximum entropy}
\subsection*{Find the probability mass function $p(x)$ that maximizes the entropy $H(X)$ of a non-negative inter-valued random variable $X$ subject to the constraint 
$$EX = \sum_{n=0}^{\infty}np(n) = A$$ 
for a fixed value $A > 0$. Evaluate this maximum $H(X)$.}

If we let $q_i = \alpha(\beta)^i$
\begin{equation}\label{}
\begin{split}
-\sum_{n=0}^{\infty}p(n) \log p(n) & \leq -\sum_{n=0}^{\infty}p(n) \log q(n) \\
& = - \log(\alpha)\sum_{n=0}^{\infty}p(n) - \log(\beta)\sum_{n=0}^{\infty}np(n) \\
& = - \log (\alpha) - A \log (\beta)
\end{split}
\end{equation}
This is true $\forall$ $\alpha$ and $\beta$ such that
$$1 = \sum_{i=0}^{\infty}\alpha\beta^i = \alpha(\frac{1}{1-\beta})$$
Also, from the question, we know that:
$$A = \sum_{i=0}^{\infty}i\alpha\beta^i = \alpha\frac{\beta}{(1-\beta)^2} $$
\begin{equation}\label{}
\begin{split}
A & = \alpha\frac{\beta}{(1-\beta)^2} \\
& = (\frac{\alpha}{1-\beta}) (\frac{\beta}{1-\beta}) \\
& = (1)(\frac{\beta}{1-\beta}) \\
& = (\frac{\beta}{1-\beta}) \\
\end{split}
\end{equation}\\
$$\Rightarrow A = (\frac{\beta}{1-\beta})$$
$$A(1-\beta) = \beta$$
$$A-A\beta = \beta$$
$$A = A\beta + \beta$$
$$A = \beta(A + 1)$$
$$\beta = \frac{A}{A+1}$$\\
$$1 = \alpha(\frac{1}{1-\beta}) = \alpha(\frac{1}{1-\frac{A}{A+1}})$$
$$\Rightarrow \alpha = 1-\frac{A}{A+1} = \frac{A+1}{A+1} - \frac{A}{A+1}$$
$$\alpha=\frac{1}{A+1}$$\\
$$H(X) = - \log \alpha - A \log \beta = (A+1)\log(A+1) - A \log A$$

\section{Uniquely decodable and instantaneous codes}
\subsection*{Let $L=\sum_{i=1}^{m} p_i l_i^{70}$ be the expected value of the 70th power of the word lengths associated with an encoding of the random variable $X$. Let $L_1 = min L$ over all instantaneous codes; and let $L_2 = min L$, over all uniquely decodable codes.  What inequality relationship exists between $L_1$ and $L_2$?}
Since all instantaneous code is by definition also uniquely decodable, we know:
$$L_1 \geq L_2$$
By the Kraft-Mcmillan Inequality we know that any uniquely decodable code must satisfy the Kraft Inequality.
$$\sum_i 2^{-l_i} \leq 1$$
We also know that given a set of codeword lengths that satisfy the Kraft inequality, it is possible to construct an instantaneous code with these codeword lengths.  Hence:
$$L_1 = L_2$$

\section{Axiomatic definition of entropy}
\subsection*{If we assume certain axioms for our measure of information, then we will be forced to use a logarithmic measure like entropy.  Shannon used this to justify his initial definition of entropy. \\If a sequence of symmetric functions $H_m(p_1, p_2,...,p_m)$ satisfies the following properties,
\begin{itemize}
\item{Normalization: $H_2(\frac{1}{2}, \frac{1}{2} )= 1$,}
\item{Continuity: $H_2(p, 1-p)$ is a continuous function of $p$,}
\item{Grouping: $H_m(p_1,p_2,...,p_m)=H_{m-1}(p_1+p_2,p_3,...,p_m)+(p_1+p_2)H_2(\frac{p_1}{p_1+p_2}, \frac{p_2}{p_1+p_2})$,}
\end{itemize}
prove that $H_m$ must be of the form
$$H_m(p_1,p_2,...,p_m)= - \sum_{i=1}^m p_i \log p_i,     m=2,3,...$$
There are various other axiomatic formulation which also result in the same definition of entropy.  See, for example, the book by Csiszar and Korner.
}
I found various papers on this proof online and was able to attempt to figure out some of the early steps that way, but I was unable to decipher the middle section, so there's a big gap where I don't understand how to put together the second half of the proof.
$$H_m(p_1,...p_m)=H_{m-k}(p_1+p_2+...p_k,p_{k+1},...,p_m)+(p_1+p_2+...+p_k)H_k(\frac{p_1}{p_1+p_2+...+p_k},...,\frac{p_k}{p_1+p_2+...+p_k})$$

\begin{equation}\label{}
\begin{split}
H_m(p_1,...p_m) & = H_{m-1}(\sum_{i=1}^2 p_i, p_3, ...,p_m)+\sum_{i=1}^2 p_i H(\frac{p_2}{\sum_{i=1}^2 p_i},1-\frac{p_2}{\sum_{i=1}^2 p_i}) \\
& = H_{m-2}(\sum_{i=1}^3 p_i, p_4, ...,p_m)+\sum_{i=1}^3 p_i H(\frac{p_3}{\sum_{i=1}^3 p_i},1-\frac{p_3}{\sum_{i=1}^3 p_i})+\sum_{i=1}^2 p_i H(\frac{p_2}{\sum_{i=1}^2 p_i},1-\frac{p_2}{\sum_{i=1}^2 p_i}) \\
& ... \\
& = H_{m-k-1}(\sum_{i=1}^k p_i,p_{k+1}..., p_m) + \sum_{i=2}^k \sum_{j=1}^i p_j H(\frac{p_i}{\sum_{j=1}^i p_j})
\end{split}
\end{equation}

$$ H_k(\frac{p_1}{\sum_{i=1}^k p_i},...,\frac{p_m}{\sum_{i=1}^k p_i}) = H_2(\frac{\sum_{i=1}^{k-1} p_i}{\sum_{i=1}^k p_i}, \frac{p_k}{\sum_{i=1}^k p_i}) = \frac{1}{\sum_{i=1}^k p_i}\sum_{i=2}^k \sum_{j=1}^i p_j $$

$$H_m(p_1,...,p_m) = H_{m-k}(\sum_{i=1}^k p_i, p_{k+1},...,p_m) + \sum_{i=1}^k p_i H_k (\frac{p_1}{\sum_{i=1}^k p_i},...,\frac{p_k}{\sum_{i=1}^k p_i})$$

Let $f(m) = H_m(\frac{1}{m}, \frac{1}{m},...,\frac{1}{m})$

\begin{equation}\label{}
\begin{split}
f(mn) & = H_{mn}(\frac{1}{mn},..., \frac{1}{mn}) \\
& = H_{mn-n}(\frac{1}{m},\frac{1}{mn},...,\frac{1}{mn}) + \frac{1}{m}H_n(\frac{1}{n},..,\frac{1}{n}) \\
& ... \\
& = H_m(\frac{1}{m},...,\frac{1}{m})+H_n(\frac{1}{n},..., \frac{1}{n}) \\
& = f(m) + f(n) \\
& \Rightarrow f(m^k)=kf(m)
\end{split}
\end{equation}

\begin{equation}\label{}
\begin{split}
f(m + 1) & = H_{m+1}(\frac{1}{m+1},..., \frac{1}{m+1}) \\
& = H(\frac{1}{m+1}, 1 - \frac{1}{m+1})+\frac{m}{m+1}H_m(\frac{1}{m},...,\frac{1}{m}) \\
& = H(\frac{1}{m+1}, 1 - \frac{1}{m+1}) + \frac{1}{m+1}f(m)
\end{split}
\end{equation}
$$H(\frac{1}{m+1}, 1 - \frac{1}{m+1}) = f(m+1)-\frac{m}{m+1}f(m)$$\\

...I can't figure out this section at all...

$$H_m(p_1,...,p_m) = - \sum_{i=1}^{m}p_i \log p_i$$





























\end{document}  